\documentclass{article}
\usepackage[magyar]{babel}
\usepackage{t1enc}
\usepackage{graphicx}
\usepackage{amsmath}
\usepackage[export]{adjustbox}
\graphicspath{ {./images/} }

\begin{document}

\section{§. Vezetők elektrosztatikus terének energiája}
Számoljuk ki töltött vezetők elektrosztatikus terének teljes $\mathcal{U}$ energiáját$^1$ :
\begin{equation} \label{eq:1}
\mathcal{U} = \frac{1}{8\pi} \int E^2 \, dV,
\end{equation}
ahol az integrált a vezetőkön kívüli térre vesszük. Transzformáljuk ezt az integrált a következőképpen:
\begin{equation} \label{eq:2}
\mathcal{U} = -\frac{1}{8\pi} \int E \, grad \, \phi \, dV = -\frac{1}{8\pi} \int div (\phi E) dV + \frac{1}{8\pi} \int \phi div E dV
\end{equation}
A második integrál eltűnik az (\ref{eq:2}) egyenlet következtében, az első pedig olyan integrállá transzformálható, amelyet az elektromos tér által határolt vezetők fellületén, valamint egy végtelen távoli felületen kell venni. Ez utóbbi integrál azonban eltűnik a végtelenben az elektromos tér eléggé gyors csökkenése következtében ($\phi$-ben a tetszőleges állandót úgy választottuk, hogy a végtelenben $\phi$ = O teljesüljön. A vezetőket a indexszel, a felületükön az állandó potenciált $\phi_a$-val jelölve kapjuk$^2$, hogy
\begin{equation} \label{eq:3}
\mathcal{U} = \frac{1}{8\pi} \sum\limits_{a} \oint \phi E_n df = \frac{1}{8\pi} \sum\limits_{a} \phi_a \oint E_n df.
\end{equation}
Végül az (1,10) összefüggésnek megfelelően bevezetve az a-adik vezetö $e_a$ össztöltését, a teljes energiára a következő kifejezést kapjuk:
\begin{equation} \label{eq:4}
\mathcal{U} = \frac{1}{2} \sum\limits_{a} e_a \phi_a ,
\end{equation}
amely analóg a ponttöltésekből álló rendszer energiájára vonatkozó kifejezéssel.

A vezetők potenciáljait és töltéseit nem lehet egyidejűleg tetszőlegesen megszabni, mert közöttük bizonyos összefüggések vannak. Mivel vákuumban a téregyenletek lineárisak és homogének, ezeknek az összefüggéseknek is lineárisaknak kell lenniük, vagyis a következő típusú egyenletekkel adhatók meg:
\begin{equation} \label{eq:5}
e_a = \sum\limits_{b} C_{ab\phi b} ,
\end{equation}
ahol a $C_{aa}$ és a $C_{ab}$ mennyiségeknek hosszúságdimenziójuk van és a vezetők relatív helyzetétől és alakjátólfüggenek. $AC_{aa}$ mennyiségeket önkapacitásoknak, a $C_{ab}(a \neq b)$ mennyiségeket elektrosztatikus indukciós együtthatóknak nevezzük. Abban a
speciális esetben, amikor csak egy vezetőnk van, $e = C\phi$ , ahol C a kapacitás. A kapacitás nagyságrendje a test lineáris méreteivel esik egybe. Az inverz összefüggések, amelyek a töltések alapián a potenciálokat adják meg:
\begin{equation} \label{eq:6}
\phi_a = \sum\limits_{b} C_{ab}^{-1} e_b ,
\end{equation}
ahol $C_{ab}^{-1}$ együtthatók mátrixa a $C_{ab}$ mátrix inverze.

Számítsuk ki vezetőrendszerek energiaváltozását, töltésük vagy potenciáljuk végtelen kis megváltozása esetén. Variálva a kiinduló (\ref{eq:1}) kifejezést, kapjuk, hogy
\begin{equation} \label{eq:7}
\delta \mathcal{U} = \frac{1}{4 \pi} \int E \delta EdV .
\end{equation}
Ezt a kifejezést két, ekvivalens módon transzformálhatjuk tovább. Ha ide az $E = -grad\phi$ kifejezést helyettesítjük és felhasználjuk azt a tényt, hogy a variált tér az eredeti térhez hasonlóan - kielégiti az (1,4) egyenletet (úgyhogy $div \delta E = 0$), akkor
\begin{equation} \label{eq:8}
\delta \mathcal{U} = \frac{1}{4 \pi} \int grad \phi * \delta EdV = \frac{1}{4\pi} \int div(\phi \delta E)dV = \frac{1}{4\pi} \sum\limits_a \phi_a \oint \delta E_n df,
\end{equation}
vagy végleges alakban
\begin{equation} \label{eq:9}
\delta \mathcal{U} = \sum\limits_a \phi_a \delta e_a ,
\end{equation}
azaz az energiaváltozást a töltésváltozásokkal kifejezve kaptuk meg. Ez az eredmény egyébként nyilvánvaló. Ez az amunka, amelyet $\delta e_a$ végtelen kis töltéseken kell végezni ahhoz, hogy azokat a végtelenből (ahol a potenciál nulla) a megadott vezetőkre vigyük.

Más részről írhatjuk, hogy
\begin{equation} \label{eq:10}
\delta \mathcal{U} = -\frac{1}{4 \pi} \int Egrad \delta \phi dV=-\frac{1}{4 \pi} \int div(E \delta \phi)dV= \frac{1}{4 \pi} \sum\limits_a \delta \phi_a \oint E_n df,
\end{equation}
vagy
\begin{equation} \label{eq:11}
\delta \mathcal{U} = \sum\limits_a e_a \delta \phi_a ,
\end{equation}
azaz, itt a vezetők potenciálváltozásaival fejeztük ki az energiaváltozást.

A (\ref{eq:9}) és (\ref{eq:10}) formulák azt mutatják, hogy az $\mathcal{U}$ energia töltés szerinti deriváltjai a vezetők potenciáljait, az $\mathcal{U}$ energia potenciálok szerinti deriváltjai pedig a töltéseket adják meg:
\begin{equation} \label{eq:12}
\frac{\partial \mathcal{U}}{\partial e_a} = \phi_a , \frac{\partial \mathcal{U}}{\partial \phi_a} = e_a .
\end{equation}
Mivel ugyanakkor a potenciálok és a töltések egymás lineáris függényei, a (\ref{eq:5}) összefüggés segítségével kapjuk, hogy
\begin{equation} \label{eq:13}
\frac{\partial^2 \mathcal{U}}{\partial \phi_a \partial \phi_b} = \frac{\partial e_b}{\partial \phi_a} = C_{ba}.
\end{equation}
Ha megváltoztatjuk a differenciálás sorrendjét, akkor az eredmény $C_{ab}$ lenne, amiből látható, hogy
\begin{equation} \label{eq:14}
C_{ab} = C_{ba},
\end{equation}
s hasonlóan $C^{-1}_{ab}=C^{-1}_{ba}$. Az $\mathcal{U}$ energiát vagy a potenciálok, vagy a töltések kvadratikus alakjaként irhatjuk fel:
\begin{equation} \label{eq:15}
\mathcal{U} = \frac{1}{2} \sum\limits_{a, b} C_{ab \phi a \phi b} = \frac{1}{2} \sum\limits_{a, b} C^{-1}_{ab} e_a e_b .
\end{equation}

Ennek a kvadratikusalaknak, mint a kiinduló (\ref{eq:1}) kifejezésnek is, pozitiv definitnek kell lennie. Ebből a feltételből különböző egyenlőtlenségeket származtathatunk, amelyeket a $C_{ab}$ együtthatóknak ki kell elégiteniük. Többek között az összes önkapacitás pozitív:
\begin{equation} \label{eq:16}
C_{aa} > 0
\end{equation}
(és $C^{-1}_{aa} > 0$).

Másrészt az összes elektrosztatikus indukciós együttható negatív:

\begin{equation} \label{eq:17}
C_{ab} < 0 \hspace{1 cm} (a \neq b )
\end{equation}
A (\ref{eq:17}) egyenlőtlenség nyilvánvaló, már az itt következő egyszerű meggondolásokból is. Tegyük fel, hogy az a vezeto kivételével minden vezetöt leföldeltünk, azaz po- tencialjuk nulla lesz. Ekkor az a-adik töltött vezetö által valamelyik b vezetön indu-
kált töltés ez = Cbala. Nyilvanvalo, hogy az indukált töltés elöjele ellenkezö lesz az indukáló potenciál elójelével és ezért Cab < O. Erröl meggyözödhetünk, ha abból in- dulunk ki, hogy az elektrosztatikus tér potenciálja a vezetökön kívül nem vehet fel maximális vagy minimális értéket. Például legyen az egyetlen nem földelt vezeto a potenciálja pozitiv. Ekkor a potenciál az egész térben pozitiv, a legkisebb (nulla)ér- téket csak földelt vezetökön éri el. Ebböl következik, hogy ezen vezetök felületén a potencial d/On normális irányú deriváltja pozitiv, és töltéseik az (1,10) összefüggés- nek megfelelöen negativak. Analóg meggondolások segitségével gyözödhetünk meg arról, hogy C=1 = 0.
\end{document}